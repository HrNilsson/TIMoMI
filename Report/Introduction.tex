%!TEX root = SSCR_project_main.tex
\chapter{Introduction}
This document describes the modeling of a Self Service Cash Register system, using VDM++ and the tool Overture.
Self Service Cash Register systems is already know in some grocery stores, and has been the inspiration for this project.
The main principle is that a customer should be able to scan and pay for his groceries, without the need for any personnel.


From the customers point of view, such a system consist of a touch screen, a barcode scanner, and some payment method, such as a credit card terminal, or a terminal for coins and bills. These parts will be modeled in an abstract way, so that the test scenarios can be understood even by a regular customer. This means that the presentation of a final model will be easier and not require that the domain expert have a big knowledge about VDM++.

\section{Purpose}
The purpose of the model is to create a simple abstract model of the system, which could be used to check the system design for errors.

The purpose of the creation of this model is also, to get a better understanding of the basic VDM++ syntax and the different constructs, that the language makes available.

\section{Abstraction}
Besides the components visible to the customer, the system also contains a connection to a global database, in which all products information and barcodes are placed. This is for simplicity abstracted away, and instead a simple instance variable is used.

Payment of groceries, using a credit card terminal, requires some internet access and pin code verification. This is also abstracted away, as to limit the size of the project.